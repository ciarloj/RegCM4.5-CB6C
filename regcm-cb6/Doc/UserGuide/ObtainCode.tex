%
% This file is part of ICTP RegCM model.
% Copyright (C) 2011 ICTP Trieste
% See the file COPYING for copying conditions.
%

\section{Simple Model User}

A packed archive file with the model code can be downloaded from:

\begin{Verbatim}
http://gforge.ictp.it/gf/project/regcm/frs
\end{Verbatim}

and it can be later on decompressed and unpacked using:

\begin{Verbatim}
$> tar -zxvf RegCM-4.4.0.tar.gz
\end{Verbatim}

\section{Model Developer}

If you plan to become a model developer, source code can be obtained via svn.
The RegCM team strongly encouragethe contributing developers to enroll on
the gforge site to always be up to date and to check on-line all the news of
the package.

\vspace{0.5cm}
\begin{tabular}{|c|}
\hline
{\bf https://gforge.ictp.it/gf/project/regcm} \\
\hline
\end{tabular}
\vspace{0.5cm}

The correct procedure is first to register on the G-forge site, then ask
the ICTP scientific team head Filippo Giorgi to be enrolled as a model
developer. After being officially granted the status, you will gain
access to the model subversion repository.

Check that {\bf Subversion} software is installed on your machine typing
the following command:

\begin{verbatim}
$> svn --version
\end{verbatim}

If your system answers \verb=command not found=, refer to your System
Administrator or software installation manual of your OS to install the
subversion software. As an example, on Scientific Linux the command
to install it as root is:

\begin{verbatim}
#> yum install subversion
\end{verbatim}

If Subversion is installed, just type the following command:

\begin{verbatim}
$> svn checkout https://gforge.ictp.it/svn/regcm/branches/regcm-core
\end{verbatim}
