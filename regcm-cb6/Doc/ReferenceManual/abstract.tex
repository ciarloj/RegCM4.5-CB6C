%%
%%   This file is part of ICTP RegCM.
%%
%%   ICTP RegCM is free software: you can redistribute it and/or modify
%%   it under the terms of the GNU General Public License as published by
%%   the Free Software Foundation, either version 3 of the License, or
%%   (at your option) any later version.
%%
%%   ICTP RegCM is distributed in the hope that it will be useful,
%%   but WITHOUT ANY WARRANTY; without even the implied warranty of
%%   MERCHANTABILITY or FITNESS FOR A PARTICULAR PURPOSE.  See the
%%   GNU General Public License for more details.
%%
%%   You should have received a copy of the GNU General Public License
%%   along with ICTP RegCM.  If not, see <http://www.gnu.org/licenses/>.
%%

\chapter{The \ac{RegCM}}

The \ac{RegCM} is a regional climate model developed throughout the years,
with a wide base of model users. It has evolved from the first version
developed in the late eighties (\ac{RegCM}1, \cite{Dickinson_89}),
\cite{Giorgi_90}), to later versions in the early nineties (\ac{RegCM}2,
\cite{Giorgi_93b}, \cite{Giorgi_93c}), late nineties (\ac{RegCM}2.5,
\cite{Giorgi_99}) and 2000s (\ac{RegCM}3, \cite{Pal_00}).

The \ac{RegCM} has been the first limited area model
developed for long term regional climate simulation, it has participated to
numerous regional model intercomparison projects, and it has been applied by
a large community for a wide range of regional climate studies, from process
studies to paleo-climate and future climate projections (\cite{Giorgi_99},
\cite{Giorgi_06}).

The \ac{RegCM} system is a community model, and in particular it is designed
for use by a varied community composed by scientists in industrialized
countries as well as developing nations (\cite{Pal_07}).

As such, it is designed to be a public, open source, user friendly and portable
code that can be applied to any region of the World. It is supported through
the Regional Climate research NETwork, or RegCNET, a widespread network of
scientists coordinated by the Earth System Physics section of
the Abdus Salam International Centre for Theoretical Physics \ac{ICTP},
being the foster the growth of advanced studies and research in developing
countries one of the main aims of the \ac{ICTP}.

The home of the model is:

\begin{center}
{\bf http://users.ictp.it/RegCNET}
\end{center}

Scientists across this network (currently subscribed by over 750 participants)
can communicate through an email list and via regular scientific workshops,
and they have been essential for the evaluation and sequential improvements of
the model.

Since the release of \ac{RegCM}3 described by \cite{Pal_07}, the model has undergone
a substantial evolution both in terms of software code and physics
representations, and this has lead to the development of a fourth version of
the model, \ac{RegCM}4, which was released by the ICTP in June 2010 as a prototype
version (\ac{RegCM}4.0) and in May 2011 as a first complete version (\ac{RegCM}4.1).

The purpose of this Manual is to provide a basic reference for \ac{RegCM}4, with
a description of the model, with a special accent to the improvements
recently introduced.
Compared to previous versions, \ac{RegCM}4 includes new land surface, planetary
boundary layer and air-sea flux schemes, a mixed convection and tropical band
configuration, modifications to the pre-existing radiative transfer and
boundary layer schemes and a full upgrade of the model code towards improved
flexibility, portability and user friendliness.

The model can be interactively coupled to a 1D lake model, a simplified aerosol
scheme (including OC, BC, SO4, dust and sea spray) and a gas phase chemistry
module (CBM-Z). Overall, \ac{RegCM}4 shows an improved performance in several
respects compared to previous versions, although further testing by the user
community is needed to fully explore its sensitivities and range of
applications.

The \ac{RegCM} is available on the World Wide Web through the ICTP Gforge
web site:

\begin{center}
  {\bf https://gforge.ictp.it/gf/project/regcm }
\end{center}
